\section*{ВВЕДЕНИЕ}
\addcontentsline{toc}{section}{ВВЕДЕНИЕ}

Вода является одним из важнейших природных ресурсов, как для поддержания жизнедеятельности биологических систем, так и для обеспечения разнообразных промышленно-технологических процессов. В настоящее время водные ресурсы активно используются в различных ключевых отраслях промышленности и жизнедеятельности человека. В металлургии вода применяется в процессах флотации - обогащения руд полиметаллов, а также для охлаждения доменных печей. Химическая промышленность активно использует воду  для очищения и охлаждения оборудования на различных нефтеперерабатывающих производствах, как компонент для создания продуктов нефтехимической отрасли, таких как пластмассы, реагенты и растворители, удобрения. Вода также активно применяется как один из материалов для создания фармацевтической продукции, такой как инъекционные растворы, препараты для наружного применения(спреи и капли).  Энергетический сектор также использует воду для охлаждения энергетического оборудования на атомных и теплоэлектростанциях. Кроме того, вода так же обеспечивает работу гидроэлектростанций. 

Загрязнение водоемов негативно сказывается на экологии, здоровье населения, препятствует развитию экономики, поэтому обеспечение экологической безопасности водоемов является одной из наиболее важных задач экологии. В настоящее время особую актуальность приобретает проблема нефтяных разливов.

По этим и другим причинам в настоящее время остро стоит вопрос сохранения чистоты водоемов. Одним из способов достижения этой цели является распознавание загрязнений на поверхности воды.

\emph{Цель настоящей работы} – разработка интеллектуальной системы для распознавания характерных пятен нефтяных разливов на поверхности водоемов. Для достижения поставленной цели необходимо решить \emph{следующие задачи:}
\begin{itemize}
\item провести анализ предметной области;
\item сформирофать датасет обучающих изображений поверхности воды;
\item разработать архитектуру нейронной сети;
\item обучить нейронную сеть;
\item провести оценку достоверности полученных результатов;
\item спроектировать настольное приложение для анализа изображений;
\item реализовать приложение, используя графический интерфейс.
\end{itemize}

\emph{Структура и объем работы.} Отчет состоит из введения, 4 разделов основной части, заключения, списка использованных источников, 2 приложений. Текст выпускной квалификационной работы изложен на \formbytotal{lastpage}{страниц}{е}{ах}{ах}.

\emph{Во введении} сформулирована цель работы, поставлены задачи разработки, описана структура работы, приведено краткое содержание каждого из разделов.

\emph{В первом разделе} на стадии анализа предметной области рассматриваются экологические аспекты загрязнения водоемов, причины и последствия разливов нефти, а также методы их распознавания.

\emph{Во втором разделе} на стадии технического задания приводятся требования к разрабатываемой интеллектуальной системе.

\emph{В третьем разделе} на стадии технического проектирования представлены проектные решения для системы.

\emph{В четвертом разделе} приводится список классов и их методов, использованных при разработке системы, производится тестирование разработанного настольного приложения.

В заключении излагаются основные результаты работы, полученные в ходе разработки.

В приложении А представлен графический материал.
В приложении Б представлены фрагменты исходного кода. 
