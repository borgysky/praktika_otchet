\addcontentsline{toc}{section}{СПИСОК ИСПОЛЬЗОВАННЫХ ИСТОЧНИКОВ}

\begin{thebibliography}{99}

	\bibitem{spill_db}  Алексеев Д.~В. Сравнительный анализ баз данных по разливам нефти и нефтепродуктов с морских судов / Д.~В.~Алексеев, А.~А.~Лентарёв // Вестник Государственного университета морского и речного флота имени адмирала С.~О.~Макарова. — 2022. — Т.~14. — №~6. — С.~891–904. DOI:~10.21821/2309-5180-2022-14-6-891-904. — Текст: непосредственный.
    \bibitem{spill_reasons} Владимиров~В.~А. Разливы нефти: причины, масштабы, последствия // Стратегия гражданской защиты: проблемы и исследования. — 2014. — №~1. — URL: https://cyberleninka.ru/article/n/razlivy-nefti-prichiny-masshtaby-posledstviya (дата обращения: 10.05.2025). – Текст~: непосредственный.
    \bibitem{radiophoto} Клименко~С.~К., Иванов~A.~Ю., Терелева~Н.~В. Пленочные загрязнения Керченского пролива по данным пятилетнего радиолокационного мониторинга: современное состояние и основные источники // Исследование Земли из космоса. –2022. – №~3. – С.~37–54. — DOI:~10.31857/S0205961422030071. – Текст~: непосредственный.
    \bibitem{nn_history}	Горбачевская~Е.~Н., Краснов~С.~С. История развития нейронных сетей  // Вестник ВУиТ.  — 2015. — №~1 (23). — URL: https://cyberleninka.ru/article/n/istoriya-razvitiya-neyronnyh-setey (дата обращения: 11.05.2025).  – Текст~: непосредственный.
	\bibitem{perceptron}Митина~О.~А., Ломовцев~П.~П. Перцептрон в задачах бинарной классификации // НАУ. — 2021. — №~66-1. — URL: https://cyberleninka.ru/article/n/pertseptron-v-zadachah-binarnoy-klassifikatsii (дата обращения: 11.05.2025). – Текст~: непосредственный.
	\bibitem{haikin_neural} Хайкин~С. Нейронные сети: полный курс / Саймон~Хайкин; пер. с~англ. Н.~Н.~Куссуль. --- 2-е изд., испр. --- М.: Вильямс, 2018. --- 1103~с. --- ISBN~978-5-907144-22-4. --- Текст: непосредственный.
	\bibitem{rostovtsev_neural} Ростовцев~В.~С. Искусственные нейронные сети: учебник. --- 5-е изд., стер. --- СПб.: Лань, 2025. --- 216~с. --- ISBN~978-5-507-50568-5. --- Текст: непосредственный.
	\bibitem{bychkov_cnn} Бычков~А.~Г., Киселёва~Т.~В., Маслова~Е.~В. Использование сверточных нейросетей для классификации изображений // Вестник Сибирского государственного индустриального университета. --- 2022. --- №~1. --- С.~15--19. --- Текст: непосредственный.
	\bibitem{godunov_cnn} Годунов~А.~И., Баланян~С.~Т., Егоров~П.~С. Сегментация изображений и распознавание объектов на основе технологии сверточных нейронных сетей // Надежность и качество сложных систем. --- 2021. --- №~3. --- С.~71--73. --- Текст: непосредственный.
	\bibitem{fowler_uml} Фаулер~М. UML. Основы. --- 3-е изд. --- СПб.: Символ-Плюс, 2015. --- 192~с. --- ISBN~978-5-93286-060-1. --- Текст: непосредственный.
	\bibitem{lutz_python1} Лутц~М. Изучаем Python. Том~1: учебное пособие / М.~Лутц. --- 5-е изд. --- М.: Вильямс, 2020. --- 832~с. --- ISBN~978-5-907144-52-1. --- Текст: непосредственный.
	\bibitem{chollet_python} Шолле~Ф. Глубокое обучение на Python / пер. с~англ. --- М.: ДМК~Пресс, 2018. --- 384~с. --- ISBN~978-5-4461-0770-4. --- Текст: непосредственный.
	\bibitem{lutz_python2} Лутц~М. Изучаем Python. Том~2 / М.~Лутц. --- 5-е изд. --- М.: Вильямс, 2020. --- 720~с. --- ISBN~978-5-907144-53-8. --- Текст: непосредственный.
	\bibitem{pointer_pytorch} Пойнтер~Я. Программируем с~PyTorch: создание приложений глубокого обучения. --- СПб.: Питер, 2020. --- 256~с. --- ISBN~978-5-4461-1677-5. --- Текст: непосредственный.
	\bibitem{bender_python} Бендер~Д. Python для анализа данных. --- М.: ДМК~Пресс, 2015. --- 482~с. --- ISBN~978-5-97060-315-4. --- Текст: непосредственный.
	\bibitem{halyapin_uml} Халяпин~Д.~Б. UML. Проектирование систем реального времени, параллельных и распределенных приложений / Д.~Б.~Халяпин. --- Москва: Озон, 2023. --- 352~с. --- ISBN~978-5-6048804-3-2. --- Текст: непосредственный.
	\bibitem{sorokin_cnn} Сорокин~А.~Б., Железняк~Л.~М., Зикеева~Е.~А. Сверточные нейронные сети: примеры реализаций: учебное пособие. --- М.: МИРЭА, 2020. --- 1~электрон. опт. диск (CD-ROM). --- Текст: непосредственный.
	\bibitem{mueller_brockmann} Мюллер-Брокманн~Й. Модульные системы в графическом дизайне. --- М.: Студия Артемия Лебедева, 2018. --- 176~с. --- ISBN~978-5-98062-140-7. --- Текст: непосредственный.
	\bibitem{hillard-packaging} Хиллард~Д. Публикация пакетов Python. Тестирование, распространение и автоматизация проектов. — М.: Бомбора, 2024. — 288с. — ISBN~978-5-04-189146-6. — Текст: непосредственный.
	\bibitem{vasiliev-python} Васильев~А.~Н. Программирование на~Python в~примерах и~задачах. — М.: Бомбора, 2021. — 384~с. — ISBN~978-5-04-103199-2. — Текст: непосредственный.
\end{thebibliography}
