\addcontentsline{toc}{section}{СПИСОК ИСПОЛЬЗОВАННЫХ ИСТОЧНИКОВ}

\begin{thebibliography}{99}

	\bibitem{spill_db}  Алексеев Д.~В. Сравнительный анализ баз данных по разливам нефти и нефтепродуктов с морских судов / Д.~В.~Алексеев, А.~А.~Лентарёв // Вестник Государственного университета морского и речного флота имени адмирала С.~О.~Макарова. — 2022. — Т.~14. — №~6. — С.~891–904. DOI:~10.21821/2309-5180-2022-14-6-891-904. — Текст: непосредственный.
    \bibitem{spill_reasons} Владимиров~В.~А. Разливы нефти: причины, масштабы, последствия // Стратегия гражданской защиты: проблемы и исследования. — 2014. — №~1. — URL: https://cyberleninka.ru/article/n/razlivy-nefti-prichiny-masshtaby-posledstviya (дата обращения: 10.05.2025). – Текст~: непосредственный.
    \bibitem{radiophoto} Клименко~С.~К., Иванов~A.~Ю., Терелева~Н.~В. Пленочные загрязнения Керченского пролива по данным пятилетнего радиолокационного мониторинга: современное состояние и основные источники // Исследование Земли из космоса. –2022. – №~3. – С.~37–54. — DOI:~10.31857/S0205961422030071. – Текст~: непосредственный.
    \bibitem{nn_history}	Горбачевская~Е.~Н., Краснов~С.~С. История развития нейронных сетей  // Вестник ВУиТ.  — 2015. — №~1 (23). — URL: https://cyberleninka.ru/article/n/istoriya-razvitiya-neyronnyh-setey (дата обращения: 11.05.2025).  – Текст~: непосредственный.
	\bibitem{perceptron}Митина~О.~А., Ломовцев~П.~П. Перцептрон в задачах бинарной классификации // НАУ. — 2021. — №~66-1. — URL: https://cyberleninka.ru/article/n/pertseptron-v-zadachah-binarnoy-klassifikatsii (дата обращения: 11.05.2025). – Текст~: непосредственный.
\end{thebibliography}
