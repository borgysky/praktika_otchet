\section*{ЗАКЛЮЧЕНИЕ}
\addcontentsline{toc}{section}{ЗАКЛЮЧЕНИЕ}

Развитие нейронных сетей и методов машинного обучения открыло широкие возможности для автоматизации задач анализа изображений. Современные технологии позволяют обрабатывать большие объемы визуальных данных и своевременно выявлять потенциальные угрозы окружающей среде.

В условиях роста количества загрязнений водоемов, вызванных разливами нефти, применение интеллектуальных систем является эффективным инструментом мониторинга. Их использование позволяет значительно облегчить процесс обнаружения и снизить его стоимость, увеличивая при этом скорость применения.

Для решения задачи распознавания пятен нефтяных разливов была разработана интеллектуальная система на основе сверточной нейронной сети архитектуры U-Net. Система реализована в виде настольного приложения с графическим интерфейсом.

Основные результаты работы:

\begin{enumerate}
\item Проведен анализ предметной области. Проведено исследование причин возникновения разливов и нейронных сетей, являющихся наиболее эффективным методом автоматического распознавания.
\item Разработана концептуальная модель интеллектуальной системы, определены основные требования к системе и аппаратному обеспечению.
\item Осуществлено проектирование интеллектуальной системы. Разработана архитектура настольного приложения и нейронной сети. Разработан пользовательский интерфейс приложения.
\item Реализована интеллектуальная система, проведено модульное и системное тестирование разработанной нейронной сети и настольного приложения.
\end{enumerate}

Все требования, объявленные в техническом задании, были полностью реализованы. Все задачи, поставленные в начале разработки проекта, были решены.

Готовый рабочий проект представлен в виде настольного приложения с графическим интерфейсом.
