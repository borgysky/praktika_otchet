\abstract{РЕФЕРАТ}

Объем работы равен \formbytotal{lastpage}{страниц}{е}{ам}{ам}. Работа содержит \formbytotal{figurecnt}{иллюстраци}{ю}{и}{й}, \formbytotal{tablecnt}{таблиц}{у}{ы}{}, \arabic{bibcount} библиографических источников и \formbytotal{числоПлакатов}{лист}{}{а}{ов} графического материала. Количество приложений – 2. Графический материал представлен в приложении А. Фрагменты исходного кода представлены в приложении Б.

Перечень ключевых слов: разливы нефти, нейронные сети, мониторинг, спутниковые снимки, распознавание объектов, информационная система, экология, загрязнение водоемов, компьютерное зрение, U-Net, интеллектуальная система, машинное обучение, обработка изображений, предобработка данных, пользовательский интерфейс.

Объектом разработки является система мониторинга водоемов для выявления загрязнений на основе изображений поверхности воды.

Целью выпускной квалификационной работы является разработка интеллектуальной системы распознавания пятен разливов нефти на поверхности водоемов.

В процессе создания системы была разработана архитектура нейронной сети, приложение с графическим интерфейсом для взаимодействия пользователей с системой, сформирован датасет, обучена и протестирована разработанная нейронная сеть.

\selectlanguage{english}
\abstract{ABSTRACT}
  
The volume of work is \formbytotal{lastpage}{page}{}{s}{s}. The work contains \formbytotal{figurecnt}{illustration}{}{s}{s}, \formbytotal{tablecnt}{table}{}{s}{s}, \arabic{bibcount} bibliographic sources and \formbytotal{числоПлакатов}{sheet}{}{s}{s} of graphic material. The number of applications is 2. The graphic material is presented in annex A. The layout of the site, including the connection of components, is presented in annex B.

List of keywords: oil spills, neural networks, monitoring, satellite imagery, object recognition, information system, ecology, water pollution, computer vision, U-Net, intelligent system, machine learning, image processing, data preprocessing, user interface.

The object of the research is the waterbody monitoring system for detectin pollution based on water surface imagery.

The goal of the work is to develop an intelligent system for oil spill detection on the surface of waterbodies.

During the development process, a neural network architecture was designed, a graphical user interface desktop application was developed for user interaction with the system, a dataset was prepared and the developed neural network architecture was trained and tested. 
\selectlanguage{russian}
