\section{Техническое задание}
\subsection{Основание для разработки}

Основанием для разработки является задание на выпускную квалификационную работу бакалавра "<Интеллектуальная система мониторинга и распознавания загрязнений водоемов">.

\subsection{Цель и назначение разработки}

Программная система предназначения для автоматического распознавания пятен нефтяных разливов на изображениях и их визуального выделения.

Пользователи должны иметь возможность загружать собственные изображения для анализа системой. Также должен быть реализован функционал сохранения полученных результатов программного анализа .

Задачами данной разработки являются:
\begin{itemize}
\item создание интеллектуальной системы распознавания на основе технологий нейронных сетей;
\item обучение созданной интеллектуальной системы для распознавания пятен нефтяных разливов;
\item разработка функционала для тестирования точности анализа полученной системы;
\item реализация настольного приложения с графическим интерфейсом для взаимодействия пользователей с системой.
\end{itemize}

\subsection{Требования пользователя к программной системе}

\subsubsection{Требования к данным программной системы}

Для обучения нейронной сети и анализа изображений на предмет наличия пятен разливов нефти программной системе требуется датасет, состоящий из изображений в формате JPEG, сохраненные в отдельной папке. Кроме того, для сохранения параметров обученной модели нейронной сети используются файлы формата .pth, содержащие веса и смещения модели. Для анализа изображений и тестирования нейронной сети также требуются файлы в формате .pth, содержащие параметры, предварительно полученные после обучения нейронной сети.

\subsubsection{Функциональные требования к программной системе}

Разработанная программа должна реализовывать следующие функциональные возможности:

\begin{itemize}
	\item обучение нейронной сети: после выбора директории с изображениями и директории сохранения полученных параметров, программа запускает процесс обучения нейронной сети и, по его завершении, сохраняет полученный набор параметров в указанную пользователем директорию;
	\item тестирование нейронной сети: после загрузки тестового изображения и предварительно сохраненного файла параметров нейронной сети, программа должна обрабатывать полученное изображение при помощи модели сети, выводить результат анализа в виде бинарного изображения, идеальный ожидаемый результат, полученный с помощью простого алгоритма обработки и метрики, оценивающие эффективность работы нейронной сети;
	\item распознавание нефтяных пятен: программа должна распознавать пятна разливов нефти на фотографиях при помощи предоставленных пользователем параметров на загруженном изображении;
	\item сохранение результатов: пользователь должен иметь возможность сохранения проанализированных изображений на жесткий диск.	
\end{itemize}

На рисунке~\ref{fig:usecase} предоставлены функциональные требования к системе, представленные в виде диаграммы прецедентов.

\begin{figure}[H]
	\centering
	\includegraphics[width=1\linewidth]{"images/сценарии использования"}
	\caption{Диаграмма прецедентов}
	\label{fig:usecase}
\end{figure}

\paragraph{Сценарий использования "<Обучение нейронной сети">}

Заинтересованные лица и их требования: пользователь желает обучить модель нейронной сети для распознавания нефтяных пятен на поверхности водоемов.

Предусловие: программа запущена, выбран режим "<Обучение">.

Постусловие: программа сохраняет файл, содержащий параметры весов нейронной сети.

Основной успешный сценарий:

\begin{enumerate}
	\item Пользователь нажимает на кнопку "<Выбрать папку">	.
	\item Программа открывает диалоговое окно выбора папки.
	\item Пользователь выбирает папку, содержащую изображения для обучения нейронной сети.
	\item Программа отображает путь до выбранной папки в специальном поле.
	\item Пользователь нажимает на кнопку "<Выбрать путь сохранения">.
	\item Программа открывает диалоговое окно выбора папки сохранения.
	\item Пользователь выбирает директорию, в которой будет сохранен файл, содержащий параметры нейронной сети.
	\item Программа отображает путь до выбранной директории сохранения в специальном поле.
	\item Пользователь нажимает на кнопку "<Обучить нейросеть">.
	\item Программа начинает процесс обучения нейронной сети на полученных изображениях.
	\item Программа отображает прогресс процесса обучения пользователю.
	\item Программа сохраняет полученный файл, содержащий параметры нейронной сети в папку, указанную пользователем, после завершения процесса обучения.
\end{enumerate}

\paragraph{Сценарий использования "<Тестирование нейронной сети">}

Заинтересованные лица и их требования: пользователь желает протестировать работу предварительно обученную нейронную сеть и узнать качество сегментации.

Предусловие: программа запущена, выбран режим "<Тестирование">, имеется файл с настройками нейронной сети.

Постусловие: программа отображает метрики качества сегментации, а также сравнение ожидаемых и полученных результатов.

Основной успешный сценарий:

\begin{enumerate}
	\item Пользователь нажимает на кнопку "<Выбрать изображение">.
	\item Программа открывает диалоговое окно выбора изображения, используемого для тестирования нейронной сети.
	\item Пользователь выбирает изображение.
	\item Программа отображает путь до выбранного изображения в специальном поле.
	\item Пользователь нажимает на кнопку "<Выбрать настройки">.
	\item Программа открывает диалоговое окно выбора файла настроек нейронной сети.
	\item Пользователь выбирает файл настроек нейронной сети.
	\item Программа отображает путь до выбранного файла настроек в специальном поле.
	\item Пользователь нажимает на кнопку "<Запустить тестирование">.
	\item Программа выполняет предварительную обработку изображения.
	\item Программа создает ожидаемую маску на основании загруженного изображения при помощи порогового алгоритма.
	\item Программа обрабатывает обработанное изображение, используя нейронную сеть с загруженными параметрами.
	\item Программа рассчитывает метрики качества сегментации изображения.
	\item Программа отображает ожидаемую и полученную нейронной сетью маски классов, а также метрики качества сегментации.
\end{enumerate}

\paragraph{Сценарий использования "<Анализ изображения">}

Заинтересованные лица и их требования: пользователь желает проанализировать изображение на наличие нефтяного разлива.

Предусловие: программа запущена, выбран режим "<Анализ изображения">, имеется файл с настройками нейронной сети.

Постусловие: программа отображает результаты анализа изображения.

Основной успешный сценарий:

\begin{enumerate}
	\item Пользователь нажимает на кнопку "<Выбрать изображение">.
	\item Программа открывает диалоговое окно выбора анализируемого изображения.
	\item Пользователь выбирает необходимое изображение.
	\item Программа отображает путь до выбранного изображения в специальном поле и выводит его в главное окно.
	\item Пользователь нажимает на кнопку "<Выбрать модель">.
	\item Программа открывает диалоговое окно выбора файла с настройками модели.
	\item Пользователь выбирает файл настроек нейронной сети.
	\item Программа отображает путь до выбранного файла в специальном поле.
	\item Пользователь нажимает на кнопку "<Анализировать">.
	\item Программа выполняет предварительную обработку изображения.
	\item Программа анализирует обработанное изображение на предмет наличия пятен нефтяных разливов с помощью нейронной сети.
	\item Программа отображает результат анализа на исходной версии изображения, загруженной пользователем в главном окне программы.
\end{enumerate}

\paragraph{Сценарий использования "<Сохранение результата анализа">}

Заинтересованные лица и их требования: пользователь желает сохранить результат анализа изображения нейронной сетью.

Предусловие: программа запущена, выбран режим "<Анализ изображения">, изображение предварительно загружено и проанализированно.

Постусловие: программа сохраняет результат обработки в виде изображения на жесткий диск.

Основной успешный сценарий:

\begin{enumerate}
	\item Пользователь нажимает на кнопку "<Сохранить результат">.
	\item Программа открывает диалоговое окно выбора папки для сохранения результата анализа.
	\item Пользователь выбирает необходимую папку.
	\item Программа сохраняет результат анализа в указанную папку в виде изображения.
	\item Программа уведомляет пользователя об успешном сохранении изображения.
\end{enumerate}

\subsubsection{Требования пользователя к интерфейсу приложения}

Приложение должно иметь следующие экраны:

\begin{enumerate}
	\item Экран "<Анализ изображения">. Основной экран, реализующий функционал распознавания пятен нефти на изображениях поверхности водоемов, где реализована возможность загрузки изображения для поиска нефтяных пятен, просмотр результата работы нейронной сети и его сохранения на жесткий диск.
	\item Экран "<Обучение">. Экран, позволяющий обучать нейронную сеть искать пятна разливов нефти на изображениях. Должен содержать возможность выбора папки, содержащей датасет для обучения, и директории сохранения файла, содержащего настройки параметров нейронной сети, полученные во время ее обучения.
	\item Экран "<Тестирование">. Экран для тестирования обученной модели, позволяющий выбрать файл настроек нейронной сети, тестовое изображение, а также отображающий исходное изображение, ожидаемый результат и фактический результат обработки тестового изображения нейронной сетью.
\end{enumerate}

\subsection{Нефункциональные требования к программной системе}

\subsubsection{Требования к надежности}

Программная система должна обеспечивать стабильную работу в различных условиях эксплуатации. В процессе работы приложения могут возникнуть следующие аварийные ситуации:
\begin{itemize}
	\item отсутствие файлов изображений в папке, выбранной как директория хранения датапака;
	\item ошибки загрузки или сохранения файлов настроек моделей или изображений из-за проблем с файловой системой;
	\item ошибки в работе программы из-за загрузки пользователем файлов некорректных форматов.
\end{itemize} 

Для предотвращения аварийных ситуаций программа должна корректно обрабатывать исключения при работе с файлами, предоставляя пользователям информативные сообщения об ошибках. В случае проблем с отсутствием прав доступа к  директории сохранения файлов, полученных в результате работы программы, программа должна открывать диалоговое окно с выбором другой директории.

\subsubsection{Требования к аппаратному обеспечению}

Для корректной работы программного продукта требуется центральный процессор с количеством ядер от 6 и выше с частотой ядра от 2.4 ГГц. Размер необходимой оперативной памяти - 8 Гб и выше. Кроме того, для отрисовки графического интерфейса требуется видеокарта с объемом графической памяти 4 Гб и выше, монитор. Наконец, для управления программой необходимы клавиатура и мышь.

Для ускорения процесса обучения нейронной сети возможно использование вычислительных ресурсов графического адаптера NVIDIA с поддержкой технологии CUDA. 

\subsubsection{Требования к программному обеспечению}

Для запуска и работы программы требуется компьютер под управлением операционной системы Windows 10 или Windows 11. При использовании графических адаптеров NVIDIA с поддержкой технологии CUDA для ускорения обучения нейронной сети необходима последняя версия драйверов соответствующего адаптера, а также программы CUDA Toolkit и cuDNN.

\subsection{Требования к оформлению документации}

Требования к стадиям разработки программ и программной документации для вычислительных машин, комплексов и систем независимо от их назначения и области применения, этапам и содержанию работ устанавливаются ГОСТ 19.102–77. и ГОСТ 34.601-90. 

Программная документация должна включать в себя:

\begin{itemize}
	\item анализ предметной области;
	\item техническое задания;
	\item технический проект;
	\item рабочий проект.
\end{itemize}
